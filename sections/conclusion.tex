% sections/conclusion.tex

\section{Conclusion}
\label{sec:conclusion}

This paper has presented \textbf{ADL~2.0}, a core specification framework for LLM-based assistants designed to unify and extend prior assistant description efforts. Building on experience gained from the application of ADL~1.0 in educational and collaborative domains, and informed by the development of tutor- and teammate-oriented specifications, ADL~2.0 distills a minimal, extensible core for assistant description.

The central contribution of ADL~2.0 lies in its explicit separation between a domain-agnostic core and domain-specific profiles. By elevating concepts such as role, collaboration, and boundaries to first-class specification elements, and by supporting declarative specialization through inheritance, ADL~2.0 addresses key limitations identified in earlier approaches. This design enables the systematic construction of diverse assistant profiles, including those exemplified by TDL and TMDL, without fragmenting the specification ecosystem.

As a specification-focused contribution, this work does not claim empirical validation of the proposed language. Instead, it articulates a set of design claims and research hypotheses that define a clear agenda for future evaluation. These hypotheses concern role consistency, boundary compliance, authoring efficiency, and the unification of domain-specific assistant descriptions under a shared core framework.

Future work will focus on the empirical assessment of these hypotheses through user studies, behavioral analyses, and tool-supported authoring evaluations. Additional directions include the development of authoring and validation tools, the exploration of formal semantics for inheritance resolution, and the extension of ADL~2.0 to additional domains beyond education and human--AI collaboration.

By providing a stable and extensible core specification, ADL~2.0 aims to support the systematic, reusable, and transparent design of LLM-based assistants, and to serve as a foundation for future research on assistant specification languages.
