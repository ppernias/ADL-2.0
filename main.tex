\documentclass[conference]{IEEEtran}

\usepackage[T1]{fontenc}
\usepackage[utf8]{inputenc}
\usepackage{cite}
\usepackage{url}
\usepackage{graphicx}
\usepackage{booktabs}
\usepackage{amsmath,amssymb}
\usepackage{enumitem}
\usepackage{hyperref}

\hypersetup{hidelinks}

\setlist[itemize]{noitemsep, topsep=2pt, leftmargin=*}

\title{ADL 2.0: A Core Specification Framework for LLM-based Assistants}

\author{
    \IEEEauthorblockN{Pedro A. Pernías Peco}
    \IEEEauthorblockA{
        Departamento de Lenguajes y Sistemas Informáticos\\
        Universidad de Alicante\\
        Alicante, España\\
        p.pernias@ua.es
    }
    \and
    \IEEEauthorblockN{M. Pilar Escobar Esteban}
    \IEEEauthorblockA{
        Departamento de Lenguajes y Sistemas Informáticos\\
        Universidad de Alicante\\
        Alicante, España\\
        pilar.escobar@ua.es
    }
}

\begin{document}
\maketitle

\begin{abstract}
The increasing deployment of Large Language Models (LLMs) in educational and collaborative settings has shifted the primary challenge from model performance to the systematic specification of assistant behavior. While early approaches relied on unstructured prompting, recent work has highlighted the need for explicit, reusable, and verifiable assistant descriptions. The Assistant Description Language (ADL~1.0) represented an initial step in this direction, and its application in concrete domains led to the development of specialized specifications such as Tutor Description Languages (TDL) and Team Mate Description Languages (TMDL).

This paper introduces \textbf{ADL~2.0}, a refactored core specification framework designed to unify and generalize prior assistant description efforts. ADL~2.0 distills a minimal, domain-agnostic core informed by lessons learned from the practical development of tutor- and teammate-oriented specifications. The language explicitly separates core assistant properties from domain-specific profiles, supports declarative specialization through inheritance, and treats operational boundaries as a mandatory, first-class component of assistant design.

As a specification-focused contribution, this paper does not present empirical evaluations. Instead, it articulates a set of design claims and research hypotheses concerning role clarity, boundary compliance, specification reuse, and the unification of domain-specific assistant descriptions under a shared core framework. By providing a stable and extensible specification foundation, ADL~2.0 aims to support the systematic, reusable, and transparent design of LLM-based assistants across educational and collaborative domains.
\end{abstract}


\begin{IEEEkeywords}
LLM, specification, ADL, TDL, TMDL
\end{IEEEkeywords}

\section{Introduction}

Large Language Models (LLMs) are increasingly used to support complex educational and collaborative activities, including tutoring, learning support, and teamwork assistance. As these systems are deployed in diverse and sensitive contexts, the primary challenge has shifted from raw model performance to the \emph{systematic specification of assistant behavior}, including roles, responsibilities, interaction patterns, and operational boundaries.

Early approaches to configuring LLM-based assistants relied primarily on unstructured prompts or ad-hoc templates. While effective for rapid prototyping, such approaches offer limited reproducibility, weak guarantees of role consistency, and poor support for reuse and maintenance. In response, structured approaches to assistant specification have emerged, most notably through the introduction of the Assistant Description Language (ADL~1.0), which provided an initial framework for describing assistants in a declarative manner.

As ADL~1.0 began to be applied in concrete domains, particularly in educational tutoring and collaborative teamwork scenarios, domain-specific extensions emerged. The Tutor Description Language (TDL) and the Team Mate Description Language (TMDL) were developed as specialized instantiations of ADL~1.0, addressing the distinct requirements of educational support and human--AI collaboration respectively. Importantly, TDL and TMDL were not conceived as independent languages, but rather as pragmatic specializations built on top of the original ADL framework. Experience with these specializations highlighted both the strengths and limitations of ADL~1.0, and motivated the need for a clearer separation between core assistant specification constructs and domain-specific concerns.

This paper introduces \textbf{ADL~2.0} as a response to these lessons learned. ADL~2.0 is not proposed as an alternative to TDL or TMDL, but rather as a \emph{refactoring and generalization of the ADL core}, informed by the practical development of domain-specific languages. The goal of ADL~2.0 is to provide a minimal, extensible, and reusable core specification framework upon which domain-specific assistant profiles---including tutors and teammates---can be systematically defined.

The design of ADL~2.0 is guided by four primary goals: (1) \emph{expressiveness}, enabling diverse assistant roles to be specified without loss of domain-specific detail; (2) \emph{modularity and reuse}, supporting inheritance and specialization through declarative mechanisms; (3) \emph{role clarity and boundary enforcement}, making constraints a first-class element of assistant design; and (4) \emph{validation and reproducibility}, through schema-based verification independent of execution engines or underlying models.

This work is positioned as a \emph{specification and design contribution}. It does not report empirical user studies or performance benchmarks. Instead, following the precedent established by prior work on TMDL, the paper articulates a set of design claims and research hypotheses concerning the expected benefits of ADL~2.0, and illustrates its applicability through representative assistant profiles derived from educational and collaborative domains. Empirical validation of these hypotheses is left for future work.

The main contributions of this paper are:
\begin{itemize}
    \item the formal definition of ADL~2.0 as a core specification language for LLM-based assistants;
    \item a principled treatment of boundaries and role constraints as first-class specification elements;
    \item a demonstration that domain-specific languages such as TDL and TMDL can be expressed as specialized profiles of ADL~2.0 without loss of expressiveness; and
    \item a research agenda outlining hypotheses and evaluation strategies for future empirical studies.
\end{itemize}

% sections/related_work.tex

\section{Related Work}
\label{sec:related_work}

\subsection{From Prompting to Structured Assistant Specification}
Early deployments of LLM-based assistants were primarily configured through free-form prompting, often combining role instructions, behavioral guidelines, and task constraints into a single prompt. While flexible, this approach makes assistant behavior difficult to reproduce, compare, and maintain across contexts. As a result, recent work has emphasized the need for more structured and declarative approaches to assistant configuration.

Several strands of work have explored structured prompting and declarative control of language model behavior. Approaches such as DSPy treat prompt design as a form of declarative programming that can be compiled and optimized \cite{khattab2023dspy}. Industrial guidelines also recommend the use of explicit structural markers (e.g., XML tags) to improve clarity and controllability of prompts \cite{anthropic2024prompting}. More recently, empirical studies have examined the reliability of structured output generation and schema-constrained responses, highlighting both their potential and limitations \cite{geng2025structured, elnashar2025promptstyles}.

Despite these advances, most structured prompting approaches remain focused on output formatting or task execution, rather than on the explicit specification of assistant roles, collaboration patterns, and operational boundaries.

\subsection{ADL and Educational Assistant Specification}
The Assistant Description Language (ADL~1.0) represents an early effort to treat assistant configuration as a first-class design artifact. ADL~1.0 proposes a declarative architecture in which pedagogical design decisions are explicitly represented and translated into automated tutoring behavior, supporting reuse and systematic assistant construction \cite{pernias2025adl}. This work demonstrates the feasibility of separating pedagogical intent from execution mechanisms, particularly in educational settings.

As ADL~1.0 was applied in practice, the need for more specialized representations became apparent. In educational contexts, tutor-oriented specifications must capture pedagogical strategies such as scaffolding, formative feedback, and controlled disclosure of solutions, while enforcing constraints related to academic integrity. These requirements motivate domain-specific tutor profiles, commonly referred to as Tutor Description Languages (TDL), which can be understood as structured specializations built upon the original ADL framework.

\subsection{Human--AI Teaming and Teammate-Oriented Specifications}
In parallel, a substantial body of literature has examined human--AI collaboration and the conditions under which AI systems function effectively as teammates rather than tools. Research highlights the importance of role clarity, shared mental models, and explicit interaction norms for effective human--AI teams \cite{bansal2019mental, seeber2020machines}. More recent surveys and conceptual frameworks emphasize boundary management and expectation alignment as central challenges in human--AI teaming \cite{lou2025humanai, gonzalez2023cohumain, gupta2025cohumain}.

These insights have motivated teammate-oriented assistant specifications, often referred to as Team Mate Description Languages (TMDL), which focus on collaboration protocols, division of responsibilities, and behavioral constraints to reduce coordination failures. While such specifications differ in emphasis from tutor-oriented descriptions, they share core structural concerns with ADL-based educational assistants, including role definition, interaction patterns, and boundary enforcement.

\subsection{Schema-Based Validation and Configuration-as-Code}
A complementary line of work treats system configuration artifacts as machine-checkable specifications. Schema-based validation is widely used in software engineering to enforce structural correctness and support tool-assisted development. Recent benchmarking work has demonstrated both the feasibility and the challenges of enforcing schema-constrained outputs in LLM-based systems \cite{geng2025structured}.

ADL~2.0 adopts a configuration-as-code perspective by defining assistant specifications through an explicit schema with required components and extensibility mechanisms. This approach supports validation independent of execution engines, and aligns with broader trends in declarative system specification while targeting assistant-specific concerns.

\subsection{Positioning of This Work}
This paper contributes at the level of \emph{language design and specification}. Rather than evaluating assistant performance or learning outcomes, we focus on unifying existing assistant description efforts under a shared core framework. Drawing on lessons learned from applying ADL~1.0 in educational and collaborative contexts, and informed by research on human--AI teaming, ADL~2.0 is proposed as a minimal and extensible specification core capable of supporting both tutor- and teammate-oriented assistant profiles.

% sections/lessons_learned.tex

\section{From ADL~1.0 to ADL~2.0: Design Lessons Learned}
\label{sec:lessons_learned}

The design of ADL~2.0 is the result of practical experience gained through the development and use of ADL~1.0 in concrete application domains, most notably educational tutoring and human--AI collaboration. In particular, the construction of domain-specific specifications such as Tutor Description Languages (TDL) and Team Mate Description Languages (TMDL) exposed a number of structural limitations in the original ADL design. This section summarizes the main lessons learned and motivates the key design decisions introduced in ADL~2.0.

\subsection{Need for a Clear Separation Between Core and Profile}
ADL~1.0 provided a unified structure for describing assistants, but in practice it blurred the distinction between core assistant properties and domain-specific concerns. As TDL and TMDL specifications were developed on top of ADL~1.0, it became increasingly difficult to determine which elements belonged to the general assistant description and which were specific to a particular domain or use case.

This lack of separation led to duplicated constructs, ad-hoc conventions, and limited reuse across assistant types. One of the primary lessons learned is the importance of explicitly distinguishing a minimal, domain-agnostic core from domain-specific profiles. ADL~2.0 addresses this issue by defining a stable set of core sections shared by all assistants, while allowing domain-specific behavior to be expressed through specialization rather than structural modification.

\subsection{Extensibility Through Declarative Specialization}
In ADL~1.0, extending an existing assistant specification often required copying and modifying large portions of the description, increasing the risk of inconsistency and error. This limitation became particularly apparent when creating families of related assistants, such as multiple tutor variants or teammates with slightly different collaboration roles.

Experience with TDL and TMDL highlighted the need for a principled mechanism to support reuse and controlled variation. ADL~2.0 introduces explicit support for declarative specialization through inheritance mechanisms, enabling assistant profiles to extend and refine existing specifications without duplicating shared structure. This design supports modular authoring and aligns assistant specification with established practices in software configuration and language design.

\subsection{Boundaries as a First-Class Design Element}
A recurring challenge in both educational and collaborative contexts is the management of assistant boundaries. In tutoring scenarios, assistants must respect constraints related to academic integrity, appropriate guidance, and learner autonomy. In collaborative settings, assistants must avoid overstepping their role, making unilateral decisions, or violating user expectations.

In ADL~1.0, such constraints were typically embedded implicitly within behavioral descriptions or prompt text, making them difficult to reason about, validate, or reuse. One of the central lessons learned from TDL and TMDL development is that boundaries must be treated as explicit, first-class elements of assistant design. ADL~2.0 therefore elevates boundary specification to a mandatory component of the core language, ensuring that constraints on assistant behavior are explicitly represented and consistently enforced across profiles.

\subsection{Specification Independent of Execution}
Another lesson learned concerns the separation between assistant specification and execution. ADL~1.0 descriptions were sometimes implicitly tied to particular execution assumptions or prompting strategies, limiting portability across models and deployment environments. This coupling complicated comparison between assistants and hindered systematic evaluation.

ADL~2.0 adopts a stricter separation between declarative specification and execution semantics. The language defines \emph{what} an assistant is intended to be, rather than \emph{how} it is implemented by a specific language model or runtime system. This separation supports reproducibility, model independence, and future empirical evaluation across different execution engines.

\subsection{Lessons Summary}
Taken together, these lessons motivate the redesign of ADL as a minimal and extensible core specification language. ADL~2.0 distills common structural elements identified across tutor- and teammate-oriented assistants, while providing explicit mechanisms for specialization, boundary definition, and validation. By addressing the limitations observed in ADL~1.0, ADL~2.0 aims to serve as a stable foundation for future assistant description languages and profiles.

% sections/adl2_spec.tex

\section{Core Specification of ADL~2.0}
\label{sec:adl2_spec}

ADL~2.0 is designed as a minimal and extensible core specification language for describing LLM-based assistants. Rather than targeting a specific application domain, ADL~2.0 defines a common structural foundation upon which domain-specific assistant profiles can be systematically constructed. This section presents the core concepts and structural elements of the language.

\subsection{Design Principles}
The specification of ADL~2.0 is guided by three overarching principles. First, the language aims to provide a \emph{domain-agnostic core}, capturing assistant properties that are common across educational, collaborative, and other application contexts. Second, ADL~2.0 emphasizes \emph{explicitness}, requiring key aspects of assistant behavior—particularly roles and boundaries—to be declared rather than implied. Third, the language supports \emph{extensibility through specialization}, enabling reuse and controlled variation without modification of the core structure.

These principles reflect lessons learned from the application of ADL~1.0 in the development of tutor- and teammate-oriented specifications, and align ADL~2.0 with established practices in declarative language and configuration design.

\subsection{Overall Structure}
An ADL~2.0 specification describes an assistant as a structured collection of sections, each addressing a distinct aspect of assistant design. At a high level, the core structure includes:

\begin{itemize}
    \item \textbf{Metadata}, capturing versioning and descriptive information;
    \item \textbf{Identity}, defining the assistant as an entity with a stable identity;
    \item \textbf{Role}, specifying the assistant's functional position and responsibilities;
    \item \textbf{Collaboration}, describing interaction patterns with users or other agents;
    \item \textbf{Knowledge and Tools}, delimiting accessible knowledge sources and operational capabilities; and
    \item \textbf{Boundaries}, explicitly constraining assistant behavior and authority.
\end{itemize}

While additional sections may be included by specialized profiles, these core components provide a shared structural vocabulary across all ADL~2.0 specifications.

\subsection{Identity and Role Specification}
The \emph{identity} component defines the assistant as a persistent conceptual entity, independent of any particular execution instance. This includes naming and descriptive elements that support clarity, reuse, and documentation.

The \emph{role} component specifies what the assistant is intended to do within a given context. Roles capture functional intent (e.g., tutor, teammate, reviewer) rather than concrete task implementations. By separating role specification from execution strategies, ADL~2.0 allows the same role definition to be instantiated across different models or deployment environments.

This explicit role declaration is central to reducing ambiguity and expectation violations, particularly in collaborative and educational settings.

\subsection{Collaboration Model}
The collaboration component defines how the assistant is expected to interact with users and, where applicable, with other agents. This includes interaction styles, turn-taking assumptions, and coordination responsibilities. Rather than prescribing dialogue policies, ADL~2.0 specifies collaboration at a conceptual level, enabling domain-specific profiles to refine interaction protocols as needed.

This abstraction supports both individual-use assistants, such as tutors, and multi-actor settings characteristic of human--AI teaming.

\subsection{Boundaries as Mandatory Specification}
A defining feature of ADL~2.0 is the treatment of boundaries as a mandatory core element. Boundaries specify what the assistant must not do, the limits of its authority, and conditions under which it should defer to human judgment or external processes.

By making boundaries explicit and structurally required, ADL~2.0 addresses limitations observed in earlier specifications where constraints were embedded implicitly in behavioral descriptions. This design choice reflects the central role of boundary management in preventing overreach, maintaining academic integrity, and supporting trustworthy human--AI collaboration.

\subsection{Extensibility Through Inheritance}
ADL~2.0 supports extensibility through declarative specialization mechanisms that allow one specification to extend another. This enables the definition of assistant families that share a common core while differing in domain-specific behavior.

Importantly, ADL~2.0 distinguishes between structural validity and semantic interpretation. The language defines the structural constraints of valid specifications, while the resolution of inheritance and the execution semantics are delegated to the implementation environment. This separation allows ADL~2.0 to remain execution-agnostic while supporting reuse and modularity.

\subsection{Schema-Based Validation}
ADL~2.0 specifications are intended to be validated against an explicit schema that defines required components and allowable structure. Schema-based validation enables early detection of incomplete or inconsistent specifications, supports tool-assisted authoring, and facilitates reproducibility across deployments.

By adopting a schema-driven approach, ADL~2.0 treats assistant descriptions as verifiable design artifacts rather than informal configuration text.

\subsection{Summary}
ADL~2.0 defines a minimal yet expressive core for assistant specification, balancing generality with explicit support for specialization. By elevating roles, collaboration models, and boundaries to first-class elements, and by supporting inheritance and schema-based validation, ADL~2.0 provides a foundation upon which domain-specific languages such as TDL and TMDL can be consistently and systematically built.

% sections/hypotheses.tex

\section{Design Claims and Research Hypotheses}
\label{sec:hypotheses}

As a specification-focused contribution, this paper does not present empirical evaluations of ADL~2.0 in deployed systems. Instead, we articulate a set of design claims and corresponding research hypotheses that are motivated by the language design and by lessons learned from prior work on assistant specification. These hypotheses are intended to guide future empirical validation efforts in educational and collaborative contexts.

\subsection{Design Claims}
The design of ADL~2.0 is based on several core claims regarding the benefits of explicit, declarative assistant specification.

First, we claim that treating assistant configuration as a structured specification artifact, rather than as unstructured prompt text, improves clarity, reuse, and maintainability. By making roles, collaboration models, and boundaries explicit, ADL~2.0 aims to reduce ambiguity in assistant behavior and facilitate systematic comparison across assistant designs.

Second, we claim that a minimal, domain-agnostic core specification can serve as a stable foundation for multiple domain-specific assistant profiles. ADL~2.0 is designed to subsume tutor- and teammate-oriented specifications, such as those exemplified by TDL and TMDL, without loss of expressiveness.

Third, we claim that explicit boundary specification is essential for trustworthy assistant behavior in both educational and collaborative settings. By elevating boundaries to a mandatory component of the core language, ADL~2.0 seeks to address common failure modes related to overreach, expectation violations, and integrity breaches.

\subsection{Research Hypotheses}
Based on these design claims, we propose the following research hypotheses for future empirical evaluation.

\paragraph{H1: Role Consistency and Expectation Alignment}
Assistants specified using ADL~2.0 will exhibit greater role consistency and fewer expectation violations than assistants configured using unstructured or minimally structured prompting approaches. This hypothesis reflects the assumption that explicit role and collaboration specifications reduce ambiguity in assistant behavior.

\paragraph{H2: Boundary Compliance}
Explicit boundary declarations in ADL~2.0 will lead to improved compliance with domain-specific constraints, such as academic integrity requirements in tutoring contexts or authority limitations in collaborative teamwork scenarios.

\paragraph{H3: Specification Reuse and Authoring Efficiency}
The use of declarative specialization mechanisms in ADL~2.0 will reduce authoring effort and specification errors when creating families of related assistants, compared to approaches that rely on copy-and-modify practices.

\paragraph{H4: Profile Unification Without Loss of Expressiveness}
Domain-specific assistant description languages, including tutor- and teammate-oriented specifications, can be expressed as profiles of ADL~2.0 without loss of expressive power. This hypothesis can be evaluated by comparing the completeness and clarity of equivalent assistant specifications expressed directly in domain-specific languages versus as ADL~2.0 profiles.

\subsection{Evaluation Outlook}
While the validation of these hypotheses is outside the scope of the present paper, they provide a concrete research agenda for future work. Potential evaluation methods include controlled comparisons of assistant behavior, user studies focusing on role clarity and expectation management, and analyses of specification authoring processes. By framing these hypotheses explicitly, ADL~2.0 positions itself as a testable and extensible specification framework rather than a fixed or closed design.


% sections/conclusion.tex

\section{Conclusion}
\label{sec:conclusion}

This paper has presented \textbf{ADL~2.0}, a core specification framework for LLM-based assistants designed to unify and extend prior assistant description efforts. Building on experience gained from the application of ADL~1.0 in educational and collaborative domains, and informed by the development of tutor- and teammate-oriented specifications, ADL~2.0 distills a minimal, extensible core for assistant description.

The central contribution of ADL~2.0 lies in its explicit separation between a domain-agnostic core and domain-specific profiles. By elevating concepts such as role, collaboration, and boundaries to first-class specification elements, and by supporting declarative specialization through inheritance, ADL~2.0 addresses key limitations identified in earlier approaches. This design enables the systematic construction of diverse assistant profiles, including those exemplified by TDL and TMDL, without fragmenting the specification ecosystem.

As a specification-focused contribution, this work does not claim empirical validation of the proposed language. Instead, it articulates a set of design claims and research hypotheses that define a clear agenda for future evaluation. These hypotheses concern role consistency, boundary compliance, authoring efficiency, and the unification of domain-specific assistant descriptions under a shared core framework.

Future work will focus on the empirical assessment of these hypotheses through user studies, behavioral analyses, and tool-supported authoring evaluations. Additional directions include the development of authoring and validation tools, the exploration of formal semantics for inheritance resolution, and the extension of ADL~2.0 to additional domains beyond education and human--AI collaboration.

By providing a stable and extensible core specification, ADL~2.0 aims to support the systematic, reusable, and transparent design of LLM-based assistants, and to serve as a foundation for future research on assistant specification languages.

% sections/appendix_example.tex

\appendices
\section{Minimal ADL~2.0 Specification Example}
\label{app:adl2_example}

This appendix provides a minimal illustrative example of an ADL~2.0 specification. The purpose of this example is not to document the full language, but to convey the overall structure and intent of an ADL~2.0 description. The complete specification, including the formal schema and extended examples, is maintained in an open repository\footnote{\url{https://github.com/ppernias/adl20}}.

\subsection{Illustrative Example}

Listing~\ref{lst:adl2_minimal} shows a simplified ADL~2.0 specification for an educational assistant profile. The example highlights the mandatory core elements of the language, including identity, role, and boundaries, while omitting domain-specific details for brevity.

\begin{figure}[ht]
\centering
\begin{minipage}{0.95\linewidth}
\small
\begin{verbatim}
adl_version: "2.0"

metadata:
  name: "Example Tutor Assistant"
  description: "Minimal ADL 2.0 example for illustration purposes"

identity:
  type: "assistant"
  persistent: true

role:
  primary_function: "educational_tutor"
  responsibilities:
    - "support learner understanding"
    - "provide formative feedback"

boundaries:
  must_not:
    - "provide full solutions to graded assignments"
    - "misrepresent its role or authority"
  defer_to_human:
    - "requests beyond defined tutoring scope"
\end{verbatim}
\end{minipage}
\caption{Minimal illustrative ADL~2.0 specification.}
\label{lst:adl2_minimal}
\end{figure}

\subsection{Discussion}
Despite its simplicity, the example illustrates several defining characteristics of ADL~2.0. First, assistant behavior is specified declaratively through explicit sections rather than embedded in unstructured prompt text. Second, the assistant's role and responsibilities are separated from operational constraints, which are captured explicitly through boundary declarations. Finally, the specification remains independent of any particular language model or execution environment.

More complex assistant profiles, including those corresponding to tutor- and teammate-oriented specifications, can be constructed by extending such core descriptions through declarative specialization mechanisms. Readers are referred to the online repository for the authoritative specification, schema definitions, and additional examples.


\bibliographystyle{IEEEtran}
\bibliography{references}

\end{document}
